\documentclass{beamer}
\usetheme{CambridgeUS}
\usepackage{tikz}
\usetikzlibrary{matrix,arrows,fit,positioning, mindmap, trees}

\usepackage[latin1]{inputenc}
\usefonttheme{professionalfonts}
\usepackage{times}
\usepackage{xmpmulti}
\usepackage{animate}
\usepackage{amsmath}
\usepackage{verbatim}
\usepackage{graphicx}
\usepackage{xcolor}
\usepackage{mathrsfs}  
\usepackage{bussproofs}
\usepackage{tikz}
\usepackage{stmaryrd}
\usepackage{xcolor}
\newcommand{\person}[1]{\textcolor{blue}{#1}}
\newcommand{\highlight}[1]{\textcolor{red}{#1}}
% \newcommand{\example}[1]{\textcolor{blue}{#1}}
\usetikzlibrary{shapes.geometric, positioning}
\graphicspath{ {./images/} }
%\usetheme{Boadilla}
%\usecolortheme{crane}
\title[Formal Languages and Automata Theory]{Formal Languages and Automata Theory}
%\subtitle{I Am Curious}
\author[Sebastian Schlesinger]{Prof. Dr.-Ing. Sebastian Schlesinger}
\institute[HWR Berlin]{Berlin School for Economics and Law}
\date{\today}
\begin{document}
 \begin{frame}
\titlepage
\end{frame}



\begin{frame}
  \frametitle{Outline}
  \tableofcontents
\end{frame}
% Slide 1: What are Formal Languages?
\section{What are Formal Languages?}
\begin{frame}
    \frametitle{What are Formal Languages?}
    \begin{itemize}
        \item \textbf{Formal language} is a set of strings formed from an alphabet $\Sigma$.
        \item A language is defined by formal rules, usually grammar or automata.
        \item Applications: parsing, compilers, coding theory, and more.
    \end{itemize}
    \pause
    \begin{block}{Example}
        Alphabet: $\Sigma = \{a, b\}$
        \begin{itemize}
            \item String: $abba$
            \item Language: $L = \{ab, ba, abb, bba\}$
        \end{itemize}
    \end{block}
\end{frame}

% Slide 2: Formal Language Definitions
\section{Formal Language Definitions}
\begin{frame}
    \frametitle{Formal Language Definitions}
    \begin{itemize}
        \item \textbf{Alphabet} ($\Sigma$): A non-empty, finite set of symbols.
        \item \textbf{String} ($w$): A finite sequence of symbols from $\Sigma$.
        \item \textbf{Language} ($L$): A set of strings over $\Sigma$.
        \item $\Sigma^*$: The set of all possible strings over $\Sigma$, including the empty string $\epsilon$.
    \end{itemize}
    \pause
    \begin{block}{Operations on Languages}
        \begin{itemize}
            \item Union: $L_1 \cup L_2$
            \item Concatenation: $L_1 L_2 = \{xy \mid x \in L_1, y \in L_2 \}$
            \item Kleene Star: $L^* = \{ w_1 w_2 \ldots w_n \mid w_i \in L, n \geq 0 \}$
        \end{itemize}
    \end{block}
\end{frame}

% Slide 3: Chomsky Hierarchy of Languages
\section{Chomsky Hierarchy}
\begin{frame}
    \frametitle{Chomsky Hierarchy of Languages}
    \begin{itemize}
        \item \textbf{Type 0}: Recursively Enumerable (Turing Machines)
        \item \textbf{Type 1}: Context-sensitive (Linear-bounded Automata)
        \item \textbf{Type 2}: Context-free (Pushdown Automata)
        \item \textbf{Type 3}: Regular (Finite Automata)
    \end{itemize}
    \pause
    \begin{block}{Important Theorem: Language Inclusion}
        \[
        \text{Regular} \subset \text{Context-Free} \subset \text{Context-Sensitive} \subset \text{Recursively Enumerable}
        \]
    \end{block}
\end{frame}

% Slide 4: Regular Languages
\section{Regular Languages}
\begin{frame}
    \frametitle{Regular Languages}
    \begin{itemize}
        \item A language is \textbf{regular} if it can be described by a regular expression.
        \item Can be accepted by a finite automaton (DFA or NFA).
        \item Closure properties:
        \begin{itemize}
            \item Union
            \item Concatenation
            \item Kleene Star
        \end{itemize}
    \end{itemize}
    \pause
    \begin{block}{Theorem: Pumping Lemma for Regular Languages}
        If $L$ is a regular language, then there exists a constant $p$ such that any string $w \in L$ with $|w| \geq p$ can be split into three parts, $w = xyz$, such that:
        \begin{itemize}
            \item $|xy| \leq p$
            \item $|y| > 0$
            \item $xy^n z \in L$ for all $n \geq 0$
        \end{itemize}
    \end{block}
\end{frame}

% Slide 5: Context-Free Languages (CFL)
\section{Context-Free Languages}
\begin{frame}
    \frametitle{Context-Free Languages (CFL)}
    \begin{itemize}
        \item Generated by context-free grammars (CFGs).
        \item Can be accepted by pushdown automata (PDA).
        \item Closure properties:
        \begin{itemize}
            \item Union
            \item Concatenation
            \item Kleene Star
        \end{itemize}
    \end{itemize}
    \pause
    \begin{block}{Theorem: Pumping Lemma for CFLs}
        If $L$ is a context-free language, there exists a constant $p$ such that any string $w \in L$ with $|w| \geq p$ can be split into five parts, $w = uvxyz$, such that:
        \begin{itemize}
            \item $|vxy| \leq p$
            \item $|vy| > 0$
            \item $uv^n x y^n z \in L$ for all $n \geq 0$
        \end{itemize}
    \end{block}
\end{frame}

% Slide 6: Context-Sensitive Languages
\section{Context-Sensitive Languages}
\begin{frame}
    \frametitle{Context-Sensitive Languages}
    \begin{itemize}
        \item Generated by context-sensitive grammars.
        \item Can be accepted by linear-bounded automata (LBA).
        \item Closure properties:
        \begin{itemize}
            \item Union
            \item Intersection
            \item Complementation
        \end{itemize}
    \end{itemize}
    \pause
    \begin{block}{Theorem: Savitch's Theorem}
        Any context-sensitive language can be decided in deterministic space $O(n^2)$.
    \end{block}
\end{frame}

% Slide 7: Recursively Enumerable Languages
\section{Recursively Enumerable Languages}
\begin{frame}
    \frametitle{Recursively Enumerable Languages}
    \begin{itemize}
        \item A language is recursively enumerable if there exists a Turing machine that can enumerate all strings in the language.
        \item Not all recursively enumerable languages are decidable.
    \end{itemize}
    \pause
    \begin{block}{Theorem: Rice's Theorem}
        Any non-trivial property of the language recognized by a Turing machine is undecidable.
    \end{block}
\end{frame}

% Slide 8: Important Theorems in Formal Languages
\section{Important Theorems}
\begin{frame}
    \frametitle{Important Theorems in Formal Languages}
    \begin{itemize}
        \item \textbf{Myhill-Nerode Theorem}: Characterizes regular languages based on the equivalence of strings.
        \item \textbf{Kleene's Theorem}: Describes equivalence between regular expressions and finite automata.
        \item \textbf{Chomsky-Schützenberger Theorem}: Every context-free language can be represented using a Dyck language.
        \item \textbf{Rice's Theorem}: Undecidability of non-trivial properties of Turing machine languages.
    \end{itemize}
\end{frame}

  \end{document}