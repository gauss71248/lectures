\documentclass{beamer}
\usetheme{CambridgeUS}
\usepackage{tikz}
\usetikzlibrary{matrix,arrows,fit,positioning, mindmap, trees}

\usepackage[latin1]{inputenc}
\usefonttheme{professionalfonts}
\usepackage{times}
\usepackage{xmpmulti}
\usepackage{animate}
\usepackage{amsmath}
\usepackage{verbatim}
\usepackage{graphicx}
\usepackage{xcolor}
\usepackage{mathrsfs}  
\usepackage{bussproofs}
\usepackage{tikz}
\usepackage{stmaryrd}
\usepackage{xcolor}
\newcommand{\person}[1]{\textcolor{blue}{#1}}
\newcommand{\highlight}[1]{\textcolor{red}{#1}}
% \newcommand{\example}[1]{\textcolor{blue}{#1}}
\usetikzlibrary{shapes.geometric, positioning}
\graphicspath{ {./images/} }
%\usetheme{Boadilla}
%\usecolortheme{crane}
\title[Mathematical Logic]{Mathematical Logic}
%\subtitle{I Am Curious}
\author[Sebastian Schlesinger]{Prof. Dr.-Ing. Sebastian Schlesinger}
\institute[HWR Berlin]{Berlin School for Economics and Law}
\date{\today}
\begin{document}
 \begin{frame}
\titlepage
\end{frame}



\begin{frame}
  \frametitle{Outline}
  \tableofcontents
\end{frame}

\section{Introduction}
\begin{frame}{Short History of Logic}
  \begin{itemize}
    \item Logic is the study of reasoning.
    \item The first recorded use of logic was by \person{Aristotle (384--322 BC)}.
    \item Aristotle explored the concept of arguing from premises to conclusions. He calls logic conclusions \highlight{syllogisms}.
    
   
  \end{itemize}
  \begin{block}{Syllogism}
    A syllogism is a statement in which certain things [the premises] are asserted, and something else [the conclusion] necessarily follows from what is asserted. By the last sentence, I mean that the premises result in the conclusion, and by that, I mean that no additional premise is required to make the conclusion unavoidable.
  \end{block}
  Example: If all humans are mortal, and Socrates is a human, then Socrates is mortal.
\end{frame}

\begin{frame}{Propositional Logic}
  \person{Boole (1815--1864)} introduced propositional logic in the 19th century.
\end{frame}

\begin{frame}{Applications of Logic}
\begin{itemize}
  \item Electronic circuit design
  \item Logic programming (e.g. Prolog)
  \item Expert systems (form of AI)
  \item Databases (e.g. SQL using first-order logic)
  \item Formal verification of software
  \end{itemize}
\end{frame}
\section{First-order logic}
\begin{frame}{Language of first-order logic}
  
    A language $\mathscr{L}$ of first-order logic consists of the following components:
    \begin{itemize}
    \item Variable symbols: $x_1, x_2, \ldots$
    \item For each $n\in\mathbb{N}$, a set of $n$-ary function symbols: $f_0, f_1, \ldots$ The 0-ary function symbols are called constant symbols.
    \item For each $n\in\mathbb{N}$, a set of $n$-ary predicate symbols: $p_0, p_1, \ldots$ The 0-ary predicate symbols are the constants $\top$ (for \textbf{true}) and $\bot$ (for \textbf{false}).
    \item special symbols: $\neg$ (negation), $\wedge$ (conjunction), $\vee$ (disjunction), $\rightarrow$ (implication), $\leftrightarrow$ (equivalence), $\forall$ (universal quantification), $\exists$ (existential quantification), and parentheses.
  \end{itemize}

  \end{frame}

  \begin{frame}{Terms}

    The set of terms of $\mathscr{L}$ is defined inductively as follows:
    \begin{itemize}
    \item Each variable is a term.
    \item If $t_1, \ldots, t_n$ are terms and $f$ is an $n$-ary function symbol, then if $f(t_1, \ldots, t_n)$ is a term.
  \end{itemize}
  \end{frame}

  \begin{frame}{Variables in terms}
    We define a function $var: \text{Terms}\rightarrow\text{Variables}$ that maps each term to the set of variables occurring in it. The function is defined as follows:
    \begin{itemize}
    \item $var(x) = \{x\}$ for each variable $x$.
    \item $var(f(t_1, \ldots, t_n)) = var(t_1)\cup\ldots\cup var(t_n)$.
  \end{itemize}
  \end{frame}
  \begin{frame}{Formulas}
    The set of formulas of $\mathscr{L}$ is defined inductively as follows:
    \begin{itemize}
    \item If $t_1, \ldots, t_n$ are terms and $p$ is an $n$-ary predicate symbol, then if $p(t_1, \ldots, t_n)$ is a formula.
    \item If $\varphi$ is a formula, then if $\neg\varphi$ is a formula.
    \item If $\varphi_1$ and $\varphi_2$ are formulas, then if $\varphi_1\wedge\varphi_2$, $\varphi_1\vee\varphi_2$, $\varphi_1\rightarrow\varphi_2$, and $\varphi_1\leftrightarrow\varphi_2$ are formulas.
    \item If $\varphi$ is a formula and $x$ is a variable, then if $\forall x.\varphi$ and $\exists x.\varphi$ are formulas.
  \end{itemize}
  An example of a formula is $\forall x. \exists y. p(x, y) \rightarrow \neg q(y)$.
  \end{frame}

  \begin{frame}{Interpretations}
    
    An interpretation $\mathcal{M}$ of $\mathscr{L}$ consists of the following components:
    \begin{itemize}
    \item A non-empty set $D$ called the domain of $\mathcal{M}$.
    \item For each $n$-ary function symbol $f$ of $\mathscr{L}$, a function $f^{\mathcal{M}}: D^n\rightarrow D$.
    \item For each $n$-ary predicate symbol $p$ of $\mathscr{L}$, a relation $p^{\mathcal{M}}\subseteq D^n$.
  \end{itemize}
  \end{frame}

  \begin{frame}{Interpretations of Terms}
    Let $\mathcal{M}$ be an interpretation for our first-order language. 
    An assignment $\sigma$ of values to variables, i.e., $\sigma: Variables\rightarrow D$. 
    
    The value of a term $t$ under $\sigma$ is denoted by $t^{\mathcal{M}}[\sigma]$ and defined as follows:
    \begin{itemize}
    \item If $t=x$ for a variable $x$, then $t^{\mathcal{M}}[\sigma]=\sigma(x)$.
    \item If $t=f(t_1, \ldots, t_n)$, then $t^{\mathcal{M}}[\sigma]=f^{\mathcal{M}}(t_1^{\mathcal{M}}[\sigma], \ldots, t_n^{\mathcal{M}}[\sigma])$.
    \end{itemize}
  \end{frame}
  \begin{frame}{Validity of Formulas under Interpretations}
    We say an assignment $\sigma$ satisfies a formula $\varphi$ under an interpretation $\mathcal{M}$, denoted by $\mathcal{M}, \sigma\models\varphi$, iff the following conditions hold:
    \begin{itemize}
    \item $\varphi=p(t_1, \ldots, t_n)$, then if $(t_1^{\mathcal{M}}[\sigma], \ldots, t_n^{\mathcal{M}}[\sigma])\in p^{\mathcal{M}}$.
    \item $\varphi=\neg\psi$, then if $\mathcal{M}, \sigma\not\models\psi$.
    \item $\varphi=\psi_1\vee\psi_2$, then if $\mathcal{M}, \sigma\models\psi_1$ or $\mathcal{M}, \sigma\models\psi_2$.
    \item $\varphi=\psi_1\wedge\psi_2$, then if $\mathcal{M}, \sigma\models\psi_1$ and $\mathcal{M}, \sigma\models\psi_2$.
    \item $\varphi=\psi_1\rightarrow\psi_2$, then if $\mathcal{M}, \sigma\models\psi_1$ implies $\mathcal{M}, \sigma\models\psi_2$.
    \item $\varphi=\psi_1\leftrightarrow\psi_2$, then if $\mathcal{M}, \sigma\models\psi_1$ if and only if $\mathcal{M}, \sigma\models\psi_2$.
    \item $\varphi=\forall x.\psi$, then if $\mathcal{M}, \sigma[x\mapsto d]\models\psi$ for all $d\in D$.
    \item $\varphi=\exists x.\psi$, then if $\mathcal{M}, \sigma[x\mapsto d]\models\psi$ for some $d\in D$.
    \end{itemize}
    A formula $\varphi$ is satisfiable if there exists an interpretation $\mathcal{M}$ and an assignment $\sigma$ such that $\mathcal{M}, \sigma\models\varphi$.
  \end{frame}


  \begin{frame}{Models}
    An interpretation $\mathcal{M}$ is a model of a formula $\varphi$, denoted by $\mathcal{M}\models\varphi$, if for all assignments $\sigma$, $\mathcal{M}, \sigma\models\varphi$.
    
    \vspace*{0.5cm}
    A formula is satisfiable if it has a model, i.e., if there exists an interpretation $\mathcal{M}$ such that $\mathcal{M}\models\varphi$.
    
    
  \end{frame}
 

  \begin{frame}{Validity}
    A formula $\varphi$ is valid if for all interpretations $\mathcal{M}$ and all assignments $\sigma$, $\mathcal{M}, \sigma\models\varphi$.
    
    We write $\models\varphi$ to denote that $\varphi$ is valid.
  \end{frame}

  \begin{frame}{Free Variables in Fomulas}
    The set of free variables of a formula $\varphi$, denoted by $FV(\varphi)$, is defined inductively as follows:
    \begin{itemize}
    \item $FV(p(t_1, \ldots, t_n))=var(t_1)\cup\ldots\cup var(t_n)$.
    \item $FV(\neg\psi)=FV(\psi)$.
    \item $FV(\psi_1\wedge\psi_2)=FV(\psi_1)\cup FV(\psi_2)$.
    \item $FV(\psi_1\vee\psi_2)=FV(\psi_1)\cup FV(\psi_2)$.
    \item $FV(\psi_1\rightarrow\psi_2)=FV(\psi_1)\cup FV(\psi_2)$.
    \item $FV(\forall x.\psi)=FV(\psi)\setminus\{x\}$.
    \item $FV(\exists x.\psi)=FV(\psi)\setminus\{x\}$.
    \end{itemize}
  \end{frame}

  \begin{frame}{Term Substitution}
    Let $\varphi$ be a formula, $x$ a variable, and $t$ a term. The formula $\varphi[t/x]$ is obtained by replacing all occurrences of $x$ in $\varphi$ by $t$. The substitution is defined inductively as follows:
    \begin{itemize}
    \item $(p(t_1, \ldots, t_n))[t/x]=p(t_1[t/x], \ldots, t_n[t/x])$.
    \item $(\neg\psi)[t/x]=\neg\psi[t/x]$.
    \item $(\psi_1\wedge\psi_2)[t/x]=\psi_1[t.x]\wedge\psi_2[t/x]$.
    \item $(\psi_1\vee\psi_2)[t/x]=\psi_1[t/x]\vee\psi_2[t/x]$.
    \item $(\psi_1\rightarrow\psi_2)[t/x]=\psi_1[t/x]\rightarrow\psi_2[t/x]$.
    \item   $(\forall y.\psi)[t/x]=\forall y.\psi[t/x]$ if $x\in FV(t)$.
    \item $(\exists y.\psi)[t/x]=\exists y.\psi[t/x]$ if $x\in FV(t)$.
    \item $(\forall x.\psi)[t/x]=\forall x.\psi$.
    \item $(\exists x.\psi)[t/x]=\exists x.\psi$.
    \end{itemize}

    So, $\varphi[t/x]$ represents the formular obtained by substituting every \textbf{free} occurrence of the variable $x$ in $\varphi$ by the term $t$.
  \end{frame}
    
  \begin{frame}{Calculus}
    A calculus is a mechanism to prove formulas by applying rules.

    A rule of a calculus has the form $\frac{\varphi_1, \ldots, \varphi_n}{\psi}$, where $\varphi_1, \ldots, \varphi_n$ are premises and $\psi$ is the conclusion. The rule states that if $\varphi_1, \ldots, \varphi_n$ are derivable, then $\psi$ is derivable.

    We denote that a formula can be proved by a calculus by $\vdash\varphi$.
  \end{frame}


  \begin{frame}{Calculus of Natural Deduction}
    \begin{itemize}
      \item inspired by the way humans reason
      \item goal is to prove a conclusion from a set of premises
      \item idea for a proof: start with the conclusion and work backwards
    \end{itemize}
  \end{frame}

  \begin{frame}{Rules of Calculus of Natural Deduction}
     
      \begin{prooftree}
      \AxiomC{$A\hspace*{1cm}B$}
       \RightLabel{$\wedge I$}
       \UnaryInfC{$A\wedge B$}
     \end{prooftree}
     \begin{columns}
      \column{0.5\textwidth}
      \begin{prooftree}
        \AxiomC{$A\wedge B$}
        \RightLabel{$\wedge E_1$}
        \UnaryInfC{$A$}
        \end{prooftree}
     \column{0.5\textwidth}
      \begin{prooftree}
        \AxiomC{$A\wedge B$}
        \RightLabel{$\wedge E_2$}
        \UnaryInfC{$B$}
        \end{prooftree}
    \end{columns}
    \begin{columns}
      \column{0.5\textwidth}
      \begin{prooftree}
        \AxiomC{$A$}
        \RightLabel{$\vee I$}
        \UnaryInfC{$A\vee B$}
        \end{prooftree}
     \column{0.5\textwidth}
      \begin{prooftree}
        \AxiomC{$A\vee B \hspace*{0.5cm}A\vdash C\hspace*{0.5cm}B\vdash C$}
        \RightLabel{$\vee E$}
        \UnaryInfC{$C$}
        \end{prooftree}
    \end{columns}
    \begin{columns}
      \column{0.5\textwidth}
      \begin{prooftree}
        \AxiomC{$A\vdash B$}
        \RightLabel{$\rightarrow I$}
        \UnaryInfC{$A\rightarrow B$}
        \end{prooftree}
     \column{0.5\textwidth}
      \begin{prooftree}
        \AxiomC{$A \hspace*{0.5cm}A\rightarrow B$}
        \RightLabel{$\rightarrow E$}
        \UnaryInfC{$B$}
        \end{prooftree}
    \end{columns}
  \end{frame}


  \begin{frame}{Rules of the Calculus of Natural Deduction}
   
  \begin{columns}
    \column{0.5\textwidth}
    \begin{prooftree}
      \AxiomC{$A\vdash B\hspace*{0.5cm} A\vdash\neg B$}
      \RightLabel{$\neg I$}
      \UnaryInfC{$\neg A$}
    \end{prooftree}
  \column{0.5\textwidth}
  \begin{prooftree}
    \AxiomC{$A\hspace*{0.5cm}\neg A$}
    \RightLabel{$\neg E$}
    \UnaryInfC{$\bot$}
  \end{prooftree}
  \end{columns}
  \begin{columns}
    \column{0.5\textwidth}
    \begin{prooftree}
    \AxiomC{$\bot$}
    \RightLabel{$\bot I$}
    \UnaryInfC{$A$}
  \end{prooftree}
  \column{0.5\textwidth}
  \begin{prooftree}
    \AxiomC{$A$}
    \RightLabel{$\neg\neg I$ ($\neg\neg E$ analogously)}
    \UnaryInfC{$\neg\neg A$}
  \end{prooftree}
  \end{columns}
  \begin{columns}
    \column{0.5\textwidth}
    \begin{prooftree}
    \AxiomC{$A[u/x]$ (for a variable $u\notin FV(A)$)}
    \RightLabel{$\forall I$}
    \UnaryInfC{$\forall x.A$}
  \end{prooftree}
  \column{0.5\textwidth}
  \begin{prooftree}
    \AxiomC{$\forall x.A$}
    \RightLabel{$\forall E$}
    \UnaryInfC{$A(t)$ for a term $t$}
  \end{prooftree}
  \end{columns}

  \begin{columns}
    \column{0.5\textwidth}
    \begin{prooftree}
    \AxiomC{$A[u/x]$ (for a variable $u\notin FV(A)$)}
    \RightLabel{$\forall I$}
    \UnaryInfC{$\forall x.A$}
  \end{prooftree}
  \column{0.5\textwidth}
  \begin{prooftree}
    \AxiomC{$\forall x.A$}
    \RightLabel{$\forall E$}
    \UnaryInfC{$A(t)$ for a term $t$}
  \end{prooftree}
  \end{columns}

  

  \end{frame}


  \begin{frame}{Rules of the Calculus of Natural Deduction}


      \begin{prooftree}
      \AxiomC{$A[t/x]$ (for a term $t$)}
      \RightLabel{$\exists I$}
      \UnaryInfC{$\exists x.A$}
    \end{prooftree}

    \begin{prooftree}
      \AxiomC{$\exists x.A\hspace*{0.5cm} A(u)\vdash B$ where $u$ is a variable not in $B$}
      \RightLabel{$\exists E$}
      \UnaryInfC{$B$}
    \end{prooftree}


    \vspace{0.5cm}
    Example: I know that there is a Norwegian logician. I was able to infer from 'Skolem was Scandinavian and Logician'. Hence, from 'there exists a Norwegian logician', I can infer 'there exists a Scandinavian logician'.

  \end{frame}

\begin{frame}{Example Proof via the Calculus of Natural Deduction}

  
  We want to show: $((p\wedge q)\rightarrow q)\rightarrow r\vdash (p\wedge q)\rightarrow r$

  \begin{prooftree}
    \AxiomC{$p\wedge q$}
    \RightLabel{$\wedge E$}
    \UnaryInfC{$q$}
    \RightLabel{$\rightarrow I$}
    \UnaryInfC{$(p\wedge q)\rightarrow q$}
    
    \AxiomC{$((p\wedge q)\rightarrow q)\rightarrow r$}
    \RightLabel{$\rightarrow E$}
    \BinaryInfC{$r$}
    \RightLabel{$\rightarrow I$}
    \UnaryInfC{$(p\wedge q)\rightarrow r$}
  \end{prooftree}
\end{frame}

\begin{frame}{Issues with the Calculus of Natural Deduction}
  \begin{itemize}
    \item The 
  \end{itemize}
\end{frame}
% ****************************



% \begin{frame}
%   \frametitle{Some Natural Deduction Rules}
%   \begin{columns}
  
%   \column{0.5\textwidth}
%   \textbf{Conjunction Introduction ($\land$ I)}
%   \begin{prooftree}
%     \AxiomC{$A$}
%     \AxiomC{$B$}
%     \RightLabel{$\land$ I}
%     \BinaryInfC{$A \land B$}
%   \end{prooftree}
  
%   \vspace{10pt}
  
%   \textbf{Conjunction Elimination ($\land$ E)}
%   \begin{prooftree}
%     \AxiomC{$A \land B$}
%     \RightLabel{$\land$ E$_1$}
%     \UnaryInfC{$A$}
%   \end{prooftree}
%   \begin{prooftree}
%     \AxiomC{$A \land B$}
%     \RightLabel{$\land$ E$_2$}
%     \UnaryInfC{$B$}
%   \end{prooftree}
  
%   \column{0.5\textwidth}
%   \textbf{Disjunction Introduction ($\lor$ I)}
%   \begin{prooftree}
%     \AxiomC{$A$}
%     \RightLabel{$\lor$ I$_1$}
%     \UnaryInfC{$A \lor B$}
%   \end{prooftree}
%   \begin{prooftree}
%     \AxiomC{$B$}
%     \RightLabel{$\lor$ I$_2$}
%     \UnaryInfC{$A \lor B$}
%   \end{prooftree}
  
%   \vspace{10pt}
  
%   \textbf{Disjunction Elimination ($\lor$ E)}
%   \begin{prooftree}
%     \AxiomC{$A \lor B$}
%     \AxiomC{$[A]$}
%     \noLine
%     \UnaryInfC{$\vdots$}
%     \noLine
%     \UnaryInfC{$C$}
%     \AxiomC{$[B]$}
%     \noLine
%     \UnaryInfC{$\vdots$}
%     \noLine
%     \UnaryInfC{$C$}
%     \RightLabel{$\lor$ E}
%     \TrinaryInfC{$C$}
%   \end{prooftree}
  
%   \end{columns}
%   \end{frame}
  
  % \begin{frame}
  % \frametitle{Natural Deduction Rules (cont.)}
  % \begin{columns}
  
  % \column{0.5\textwidth}
  % \textbf{Implication Introduction ($\implies$ I)}
  % \begin{prooftree}
  %   \AxiomC{$[A]$}
  %   \noLine
  %   \UnaryInfC{$\vdots$}
  %   \noLine
  %   \UnaryInfC{$B$}
  %   \RightLabel{$\implies$ I}
  %   \UnaryInfC{$A \implies B$}
  % \end{prooftree}
  
  % \vspace{10pt}
  
  % \textbf{Implication Elimination (Modus Ponens, $\implies$ E)}
  % \begin{prooftree}
  %   \AxiomC{$A \implies B$}
  %   \AxiomC{$A$}
  %   \RightLabel{$\implies$ E}
  %   \BinaryInfC{$B$}
  % \end{prooftree}
  
  % \column{0.5\textwidth}
  % \textbf{Negation Introduction ($\neg$ I)}
  % \begin{prooftree}
  %   \AxiomC{$[A]$}
  %   \noLine
  %   \UnaryInfC{$\vdots$}
  %   \noLine
  %   \UnaryInfC{$\bot$}
  %   \RightLabel{$\neg$ I}
  %   \UnaryInfC{$\neg A$}
  % \end{prooftree}
  
  % \vspace{10pt}
  
  % \textbf{Negation Elimination ($\neg$ E)}
  % \begin{prooftree}
  %   \AxiomC{$A$}
  %   \AxiomC{$\neg A$}
  %   \RightLabel{$\neg$ E}
  %   \BinaryInfC{$\bot$}
  % \end{prooftree}
  
  % \end{columns}
  % \end{frame}
  
  % \begin{frame}
  % \frametitle{Natural Deduction Rules (cont.)}
  % \begin{columns}
  
  % \column{0.5\textwidth}
  % \textbf{Double Negation Elimination ($\neg\neg$ E)}
  % \begin{prooftree}
  %   \AxiomC{$\neg\neg A$}
  %   \RightLabel{$\neg\neg$ E}
  %   \UnaryInfC{$A$}
  % \end{prooftree}
  
  % \vspace{10pt}
  
  % \textbf{Biconditional Introduction ($\iff$ I)}
  % \begin{prooftree}
  %   \AxiomC{$A \implies B$}
  %   \AxiomC{$B \implies A$}
  %   \RightLabel{$\iff$ I}
  %   \BinaryInfC{$A \iff B$}
  % \end{prooftree}
  
  % \column{0.5\textwidth}
  % \textbf{Biconditional Elimination ($\iff$ E)}
  % \begin{prooftree}
  %   \AxiomC{$A \iff B$}
  %   \RightLabel{$\iff$ E$_1$}
  %   \UnaryInfC{$A \implies B$}
  % \end{prooftree}
  % \begin{prooftree}
  %   \AxiomC{$A \iff B$}
  %   \RightLabel{$\iff$ E$_2$}
  %   \UnaryInfC{$B \implies A$}
  % \end{prooftree}
  
  % \end{columns}
  % \end{frame}

  % \begin{frame}
  %   \frametitle{Some Natural Deduction Rules: Quantifiers}
  %   \begin{columns}
    
  %   \column{0.5\textwidth}
  %   \textbf{Universal Introduction ($\forall$ I)}
  %   \begin{prooftree}
  %     \AxiomC{$[x]$}
  %     \noLine
  %     \UnaryInfC{$\vdots$}
  %     \noLine
  %     \UnaryInfC{$A(x)$}
  %     \RightLabel{$\forall$ I}
  %     \UnaryInfC{$\forall x \, A(x)$}
  %   \end{prooftree}
    
  %   \vspace{10pt}
    
  %   \textbf{Universal Elimination ($\forall$ E)}
  %   \begin{prooftree}
  %     \AxiomC{$\forall x \, A(x)$}
  %     \RightLabel{$\forall$ E}
  %     \UnaryInfC{$A(t)$}
  %   \end{prooftree}
    
  %   \column{0.5\textwidth}
  %   \textbf{Existential Introduction ($\exists$ I)}
  %   \begin{prooftree}
  %     \AxiomC{$A(t)$}
  %     \RightLabel{$\exists$ I}
  %     \UnaryInfC{$\exists x \, A(x)$}
  %   \end{prooftree}
    
  %   \vspace{10pt}
    
  %   \textbf{Existential Elimination ($\exists$ E)}
  %   \begin{prooftree}
  %     \AxiomC{$\exists x \, A(x)$}
  %     \AxiomC{$[A(x)]$}
  %     \noLine
  %     \UnaryInfC{$\vdots$}
  %     \noLine
  %     \UnaryInfC{$C$}
  %     \RightLabel{$\exists$ E}
  %     \BinaryInfC{$C$}
  %   \end{prooftree}
    
  %   \end{columns}
  %   \end{frame}

    % \begin{frame}
    %   \frametitle{Example Deduction: \(\forall x (P(x) \rightarrow Q(x)), \forall x P(x) \vdash \forall x Q(x)\)}
      
    %   \begin{prooftree}
    %     \AxiomC{}
    %     \RightLabel{Premise}
    %     \UnaryInfC{$\forall x (P(x) \rightarrow Q(x))$}
    %     \AxiomC{}
    %     \RightLabel{Premise}
    %     \UnaryInfC{$\forall x P(x)$}
    %     \RightLabel{$\forall$ E}
    %     \UnaryInfC{$P(a)$}
    %     \RightLabel{$\forall$ E}
    %     \UnaryInfC{$P(a) \rightarrow Q(a)$}
    %     \RightLabel{$\rightarrow$ E}
    %     \BinaryInfC{$Q(a)$}
    %     \RightLabel{$\forall$ I}
    %     \UnaryInfC{$\forall x Q(x)$}
    %   \end{prooftree}
      
      % \end{frame}
%*****************

\begin{frame}{Sequent Calculus}
  

  In sequent calculus, we have sequences $\Gamma\vdash\Delta$, where $\Gamma$ and $\Delta$ are sets of formulas.
  \vspace*{1cm}

  The interpretation is that if all formulas in $\Gamma$ are true, then at least one formula in $\Delta$ is true.
  \end{frame}
  
  
    \begin{frame}{Sequent Calculus Rules}
      \begin{columns}
        \column{0.5\textwidth}
      \begin{prooftree}
        \AxiomC{-}
        \RightLabel{Taut}
        \UnaryInfC{$\Gamma, \varphi \Rightarrow \varphi, \Delta$}
      \end{prooftree}
      \column{0.5\textwidth}
      \begin{prooftree}
        \AxiomC{$\Gamma\Rightarrow\Delta,\varphi\hspace*{0.5cm}\varphi,\Pi\Rightarrow\Lambda$}
        \RightLabel{Cut}
        \UnaryInfC{$\Gamma,\Pi \Rightarrow\Delta,\Lambda$}
      \end{prooftree}
    \end{columns}

      \begin{columns}
    
      \column{0.5\textwidth}
      \begin{prooftree}
        \AxiomC{-}
        \RightLabel{$\bot \Rightarrow$}
        \UnaryInfC{$\Gamma, \bot \Rightarrow \Delta$}
      \end{prooftree}
      \column{0.5\textwidth}
      \begin{prooftree}
        \AxiomC{-}
        \RightLabel{$\Rightarrow\top$}
        \UnaryInfC{$\Gamma\Rightarrow \Delta,\top$}
      \end{prooftree}
    \end{columns}


    \begin{columns}
      \column{0.5\textwidth}
    \begin{prooftree}
      \AxiomC{$\Gamma\Rightarrow\Delta$}
      \RightLabel{Weakening left}
      \UnaryInfC{$\varphi,\Gamma\Rightarrow\Delta$}
    \end{prooftree}
    \column{0.5\textwidth}
    \begin{prooftree}
      \AxiomC{$\Gamma\Rightarrow\Delta$}
      \RightLabel{Weakening right}
      \UnaryInfC{$\Gamma\Rightarrow\Delta,\varphi$}
    \end{prooftree}
  \end{columns}

  \begin{columns}
    \column{0.5\textwidth}
  \begin{prooftree}
    \AxiomC{$\varphi,\varphi,\Gamma\Rightarrow\Delta$}
    \RightLabel{Contraction left}
    \UnaryInfC{$\varphi,\Gamma\Rightarrow\Delta$}
  \end{prooftree}
  \column{0.5\textwidth}
  \begin{prooftree}
    \AxiomC{$\Gamma\Rightarrow\Delta,\varphi,\varphi$}
    \RightLabel{Contraction right}
    \UnaryInfC{$\Gamma\Rightarrow\Delta,\varphi$}
  \end{prooftree}
\end{columns}

\begin{columns}
  \column{0.5\textwidth}
\begin{prooftree}
  \AxiomC{$\Gamma,\varphi,\psi,\Pi\Rightarrow\Delta$}
  \RightLabel{Exchange left}
  \UnaryInfC{$\Gamma,\psi,\varphi,\Pi\Rightarrow\Delta$}
\end{prooftree}
\column{0.5\textwidth}
\begin{prooftree}
  \AxiomC{$\Gamma\Rightarrow\Delta,\varphi,\psi,\Lambda$}
  \RightLabel{Exchange right}
  \UnaryInfC{$\Gamma\Rightarrow\Delta,\psi,\varphi,\Lambda$}
\end{prooftree}
\end{columns}

  
  \end{frame}
  \begin{frame}{Sequent Calculus Rules}

    \begin{columns}
  
    \column{0.5\textwidth}
    \begin{prooftree}
      \AxiomC{-}
      \RightLabel{$\bot \Rightarrow$}
      \UnaryInfC{$\Gamma, \bot \Rightarrow \Delta$}
    \end{prooftree}
    \column{0.5\textwidth}
    \begin{prooftree}
      \AxiomC{-}
      \RightLabel{$\Rightarrow\top$}
      \UnaryInfC{$\Gamma\Rightarrow \Delta,\top$}
    \end{prooftree}
  \end{columns}


    \begin{columns}
    
      \column{0.5\textwidth}
      \begin{prooftree}
        \AxiomC{$\Gamma\Rightarrow\Delta,\varphi$}
        \RightLabel{$\neg\Rightarrow$}
        \UnaryInfC{$\Gamma,\neg \varphi\Rightarrow \Delta$}
      \end{prooftree}
      \column{0.5\textwidth}
      \begin{prooftree}
        \AxiomC{$\Gamma,\varphi\Rightarrow\Delta$}
        \RightLabel{$\Rightarrow\neg$}
        \UnaryInfC{$\Gamma\Rightarrow \Delta,\neg\varphi$}
      \end{prooftree}
      \end{columns}
      

      \begin{columns}
    
        \column{0.5\textwidth}
        \begin{prooftree}
          \AxiomC{$\Gamma,\varphi\Rightarrow\Delta \hspace*{0.5cm} \Gamma,\psi\Rightarrow \Delta$}
          \RightLabel{$\vee\Rightarrow$}
          \UnaryInfC{$\Gamma,\varphi\vee \psi\Rightarrow \Delta$}
        \end{prooftree}
        \column{0.5\textwidth}
        \begin{prooftree}
          \AxiomC{$\Gamma\Rightarrow\Delta,\varphi,\psi$}
          \RightLabel{$\Rightarrow\vee$}
          \UnaryInfC{$\Gamma\Rightarrow \Delta,\varphi\vee \psi$}
        \end{prooftree}
        \end{columns}

        \begin{columns}
    
          \column{0.5\textwidth}
          \begin{prooftree}
            \AxiomC{$\Gamma,\varphi,\psi\Rightarrow\Delta$}
            \RightLabel{$\wedge\Rightarrow$}
            \UnaryInfC{$\Gamma,\varphi\wedge \psi\Rightarrow \Delta$}
          \end{prooftree}
          \column{0.5\textwidth}
          \begin{prooftree}
            \AxiomC{$\Gamma\Rightarrow\Delta,\varphi \hspace*{0.5cm}\Gamma\Rightarrow\Delta,\psi$}
            \RightLabel{$\Rightarrow\wedge$}
            \UnaryInfC{$\Gamma\Rightarrow \Delta,\varphi\wedge \psi$}
          \end{prooftree}
          \end{columns}


          \begin{columns}
    
            \column{0.5\textwidth}
            \begin{prooftree}
              \AxiomC{$\Gamma\Rightarrow\Delta,\varphi\hspace*{0.5cm}\psi,\Pi\Rightarrow\Lambda$}
              \RightLabel{$\to\Rightarrow$}
              \UnaryInfC{$\varphi\to\psi,\Gamma,\Pi\Rightarrow \Delta,\Lambda$}
            \end{prooftree}
            \column{0.5\textwidth}
            \begin{prooftree}
              \AxiomC{$\Gamma,\varphi\Rightarrow\Delta,\psi$}
              \RightLabel{$\Rightarrow\to$}
              \UnaryInfC{$\Gamma\Rightarrow \Delta,\varphi\to \psi$}
            \end{prooftree}
            \end{columns}



      \end{frame}
      \begin{frame}{Sequent Calculus Rules}
        \begin{columns}
    
          \column{0.5\textwidth}
          \begin{prooftree}
            \AxiomC{$\Gamma,\varphi[t/x]\Rightarrow\Delta$}
            \RightLabel{$\forall\Rightarrow$}
            \UnaryInfC{$\Gamma,\forall x.\varphi(x)\Rightarrow\Delta$}
          \end{prooftree}
          \column{0.5\textwidth}
          \begin{prooftree}
            \AxiomC{$\Gamma\Rightarrow\Delta,\varphi[y/x]$}
            \RightLabel{$\Rightarrow\forall$}
            \UnaryInfC{$\Gamma\Rightarrow\Delta,\forall x.\varphi(x)$}
          \end{prooftree}
          \end{columns}

          
          \begin{columns}
    
            \column{0.5\textwidth}
            \begin{prooftree}
              \AxiomC{$\Gamma,\varphi[y/x]\Rightarrow\Delta$}
              \RightLabel{$\exists\Rightarrow$}
              \UnaryInfC{$\Gamma,\exists x.\varphi(x)\Rightarrow\Delta$}
            \end{prooftree}
            \column{0.5\textwidth}
            \begin{prooftree}
              \AxiomC{$\Gamma\Rightarrow\Delta,\exists x.\varphi(x),\varphi[t/x]$}
              \RightLabel{$\Rightarrow\exists$}
              \UnaryInfC{$\Gamma\Rightarrow\Delta,\exists x.\varphi(x)$}
            \end{prooftree}
            \end{columns}

            \vspace{0.5cm}

            In the quantifier rules, $t$ is a term, and $y$ is a 'fresh' variable, i.e., a variable that does not occur in $\Gamma$, $\Delta$, or $\varphi$. 
            
            Alternatively, the rules can also be stated in the form
        
            \begin{prooftree}
              \AxiomC{$\Gamma\Rightarrow\Delta,\varphi$}
              \RightLabel{$\Rightarrow\forall$}
              \UnaryInfC{$\Gamma\Rightarrow\Delta,\forall x.\varphi(x)$}
            \end{prooftree}

            Here, it must be guaranteed that $x$ is not free in any formula in $\Gamma$ or $\Delta$. The existential formula can be handled similarly.
      \end{frame}
      
    %   \begin{frame}
    %   \frametitle{Sequent Calculus Rules: Logical Connectives (cont.)}
    %   \begin{columns}
      
    %   \column{0.5\textwidth}
    %   \textbf{Implication Rules ($\implies$)}
    %   \begin{prooftree}
    %     \AxiomC{$\Gamma, A \vdash B$}
    %     \RightLabel{$\implies$ R}
    %     \UnaryInfC{$\Gamma \vdash A \implies B$}
    %   \end{prooftree}
      
    %   \begin{prooftree}
    %     \AxiomC{$\Gamma \vdash A$}
    %     \AxiomC{$\Gamma, B \vdash \Delta$}
    %     \RightLabel{$\implies$ L}
    %     \BinaryInfC{$\Gamma, A \implies B \vdash \Delta$}
    %   \end{prooftree}
      
    %   \column{0.5\textwidth}
    %   \textbf{Negation Rules ($\neg$)}
    %   \begin{prooftree}
    %     \AxiomC{$\Gamma, A \vdash \Delta$}
    %     \RightLabel{$\neg$ R}
    %     \UnaryInfC{$\Gamma \vdash \neg A$}
    %   \end{prooftree}
      
    %   \begin{prooftree}
    %     \AxiomC{$\Gamma \vdash A$}
    %     \RightLabel{$\neg$ L}
    %     \UnaryInfC{$\Gamma, \neg A \vdash \Delta$}
    %   \end{prooftree}
      
    %   \end{columns}
    %   \end{frame}
      
    %   \begin{frame}
    %   \frametitle{Sequent Calculus Rules: Quantifiers}
    %   \begin{columns}
      
    %   \column{0.5\textwidth}
    %   \textbf{Universal Quantifier ($\forall$)}
    %   \begin{prooftree}
    %     \AxiomC{$\Gamma \vdash A(x)$}
    %     \RightLabel{$\forall$ R}
    %     \UnaryInfC{$\Gamma \vdash \forall x \, A(x)$}
    %   \end{prooftree}
      
    %   \begin{prooftree}
    %     \AxiomC{$\Gamma, A(t) \vdash \Delta$}
    %     \RightLabel{$\forall$ L}
    %     \UnaryInfC{$\Gamma, \forall x \, A(x) \vdash \Delta$}
    %   \end{prooftree}
      
    %   \column{0.5\textwidth}
    %   \textbf{Existential Quantifier ($\exists$)}
    %   \begin{prooftree}
    %     \AxiomC{$\Gamma \vdash A(t)$}
    %     \RightLabel{$\exists$ R}
    %     \UnaryInfC{$\Gamma \vdash \exists x \, A(x)$}
    %   \end{prooftree}
      
    %   \begin{prooftree}
    %     \AxiomC{$\Gamma, A(x) \vdash \Delta$}
    %     \RightLabel{$\exists$ L}
    %     \UnaryInfC{$\Gamma, \exists x \, A(x) \vdash \Delta$}
    %   \end{prooftree}
      
    %   \end{columns}
    %   \end{frame}
      
    %   \begin{frame}
    %   \frametitle{Sequent Calculus Rules: Structural Rules}
    %   \begin{columns}
      
    %   \column{0.5\textwidth}
    %   \textbf{Weakening}
    %   \begin{prooftree}
    %     \AxiomC{$\Gamma \vdash \Delta$}
    %     \RightLabel{WL}
    %     \UnaryInfC{$\Gamma, A \vdash \Delta$}
    %   \end{prooftree}
    %   \begin{prooftree}
    %     \AxiomC{$\Gamma \vdash \Delta$}
    %     \RightLabel{WR}
    %     \UnaryInfC{$\Gamma \vdash \Delta, A$}
    %   \end{prooftree}
      
    %   \vspace{10pt}
      
    %   \textbf{Contraction}
    %   \begin{prooftree}
    %     \AxiomC{$\Gamma, A, A \vdash \Delta$}
    %     \RightLabel{CL}
    %     \UnaryInfC{$\Gamma, A \vdash \Delta$}
    %   \end{prooftree}
    %   \begin{prooftree}
    %     \AxiomC{$\Gamma \vdash \Delta, A, A$}
    %     \RightLabel{CR}
    %     \UnaryInfC{$\Gamma \vdash \Delta, A$}
    %   \end{prooftree}
      
    %   \column{0.5\textwidth}
    %   \textbf{Exchange}
    %   \begin{prooftree}
    %     \AxiomC{$\Gamma, A, B, \Sigma \vdash \Delta$}
    %     \RightLabel{EL}
    %     \UnaryInfC{$\Gamma, B, A, \Sigma \vdash \Delta$}
    %   \end{prooftree}
    %   \begin{prooftree}
    %     \AxiomC{$\Gamma \vdash \Delta, A, B, \Pi$}
    %     \RightLabel{ER}
    %     \UnaryInfC{$\Gamma \vdash \Delta, B, A, \Pi$}
    %   \end{prooftree}
      
    %   \vspace{10pt}
      
    %   \textbf{Cut}
    %   \begin{prooftree}
    %     \AxiomC{$\Gamma \vdash A, \Delta$}
    %     \AxiomC{$\Gamma, A \vdash \Delta$}
    %     \RightLabel{Cut}
    %     \BinaryInfC{$\Gamma \vdash \Delta$}
    %   \end{prooftree}
      
    %   \end{columns}
    %   \end{frame}

      

        \begin{frame}
          \frametitle{Example Deduction: \(\forall x.(P(x)\wedge Q\Rightarrow\forall x.P(x)\)}
          
          \begin{prooftree}
            \AxiomC{}
            \RightLabel{Taut}
            \UnaryInfC{$P(x),Q\Rightarrow P(x)$}
          
            
            \RightLabel{$\wedge\Rightarrow$ }
            \UnaryInfC{$P(x)\wedge Q\Rightarrow P(x)$}
          
            \RightLabel{$\forall\Rightarrow$}
            \UnaryInfC{$\forall x.(P(x\wedge Q)\Rightarrow P(x))$}
          
            \RightLabel{$\Rightarrow\forall$}
            \UnaryInfC{$\forall x.(P(x)\wedge Q)\Rightarrow\forall x.P(x)$}
          
          \end{prooftree}
          \vspace*{0.5cm}
          Here, $\forall\Rightarrow$ uses $[x/x]$ as replacement, i.e., just the same free variable is taken.
          \end{frame}


          \begin{frame}
            \frametitle{Example Deduction: \(\forall x.(A\to B)\Rightarrow A\to\forall x.B\)}
            
            \begin{prooftree}
              \AxiomC{}
              \RightLabel{Taut}
              \UnaryInfC{$A\Rightarrow A,B$}

              \AxiomC{}
              \RightLabel{Taut}
              \UnaryInfC{$A,B\Rightarrow B$}
            
              
              
              \RightLabel{$\to\Rightarrow$ }
              \BinaryInfC{$A,A\to B\Rightarrow B$}
            
             
              \RightLabel{$\forall\Rightarrow$}
              \UnaryInfC{$A,\forall x.(A\to B)\Rightarrow B$}
            
              \RightLabel{$\Rightarrow\forall$}
              \UnaryInfC{$A,\forall x.(A\to B)\Rightarrow \forall x.B$}
              \RightLabel{$\Rightarrow\to$}
              \UnaryInfC{$\forall x.(A\to B)\Rightarrow A\to \forall x.B$}
              
            
            \end{prooftree}
            \vspace*{0.5cm}
            Here, the application of $\Rightarrow\forall$ requires that $x$ is not free in $A$.

            \end{frame}

            \begin{frame}
              \frametitle{Example of a Failing Deduction: \(\exists x.P(x)\wedge \exists x.Q(x)\Rightarrow\exists x.(P(x)\wedge Q(x))\)}
              
              \begin{prooftree}
                \AxiomC{$P(x),Q(y)\Rightarrow P(x)\wedge Q(x)$}
                \RightLabel{$\Rightarrow\exists$}
                \UnaryInfC{$P(x),Q(y)\Rightarrow\exists x.(P(x)\wedge Q(x))$}
                \RightLabel{$\exists\Rightarrow$}
                \UnaryInfC{$P(x),\exists x.Q(x)\Rightarrow\exists x.(P(x)\wedge Q(x))$}
                \RightLabel{$\exists\Rightarrow$}
                \UnaryInfC{$\exists x.P(x),\exists x.Q(x)\Rightarrow\exists x.(P(x)\wedge Q(x))$}
                \RightLabel{$\wedge\Rightarrow$}
                \UnaryInfC{$\exists x.P(x)\wedge \exists x.Q(x)\Rightarrow\exists x.(P(x)\wedge Q(x))$}
              
              \end{prooftree}
              \vspace*{0.5cm}
              Here, the deduction fails because the variable $x$ is not fresh in the application of $\exists\Rightarrow$ and therefore the new variable $y$ is introduced.
              However, then the deduction cannot be completed.
  
              \end{frame}

%*****************
        \begin{frame}{Soundness and Completeness of Sequent Calculus}
        \begin{itemize}
        \item A calculus is sound if all provable formulas are valid, denoted by $\vdash\varphi\Rightarrow\models\varphi$.
        \item A calculus is complete if all valid formulas are provable, denoted by $\models\varphi\Rightarrow\vdash\varphi$.
        \end{itemize}

        \vspace{0.5cm}
        The sequent calculus is sound and complete for first-order logic, i.e.,
        \begin{itemize}
        \item $\vdash\Gamma\Rightarrow\Delta$, then $\models\Gamma\Rightarrow\Delta$.
        \item $\models\Gamma\Rightarrow\Delta$, then $\models\Gamma\Rightarrow\Delta$.
        \end{itemize}
        \vspace*{0.5cm}

        Proof of soundness by induction on the structure of the formulas.
        Proof of completeness by constructing a Herbrand model.
      \end{frame}

      \begin{frame}{Goedel's Incompleteness Theorem}

        \begin{itemize}
        \item Goedel's first incompleteness theorem states that in any consistent formal system that is powerful enough to express arithmetic, there are true statements that cannot be proven.
        \item Goedel's second incompleteness theorem states that in any consistent formal system that is powerful enough to express arithmetic, the system cannot prove its own consistency.
        \end{itemize}
       \end{frame}

       \begin{frame}{Rice's Theorem}
        \begin{itemize}
        \item Rice's theorem states that for any non-trivial property of partial functions, i.e., a property that is not true for all partial functions or not true for none, there is no algorithm that can decide whether a given program has that property.
        \item A property is non-trivial if there are two partial functions that are computable and one has the property and the other does not.
        \end{itemize}
        \end{frame}

        \begin{frame}{Halting Problem}
          The halting problem is the problem of determining, given a program and an input, whether the program will eventually halt when run with that input.
          
          The halting problem is undecidable, i.e., there is no algorithm that can decide whether a given program halts on a given input.
        \end{frame}

        \begin{frame}{Some Conclusions}
          \begin{itemize}
            \item There are properties of programs that cannot be decided by an algorithm. 
            \item There are properties of programs that cannot be verified by a formal system.
            \item It is impossible to generally prove behavioural equivalence of programs.
          \end{itemize}
        
          
        \end{frame}
\section{Semantics of Programs}


\begin{frame}{Semantics of programs}

  There are three main types of semantics for programs:
  \begin{itemize}
  \item Operational semantics: Describes the execution of programs.
  \item Denotational semantics: Describes the meaning of programs (as a mathematical mapping of states).
  \item Axiomatic semantics: Describes properties of programs.
  \end{itemize}
  \end{frame}
  \begin{frame}{The while language}
    The while language is a simple imperative programming language with the following constructs:
    \begin{itemize}
      \item Arithmetic expressions: $E::=n\ |\ x\ |\ E+E\ |\ E-E\ |\ E*E\ |\ E/E$ (i.e., terms), where $n$ is a number and $x$ is a variable.
      \item Boolean expressions: $B::=\text{true}\ |\ \text{false}\ |\ E=E\ |\ E<E\ |\ E\leq E\ |\ \text{not } B\ |\ B \text{ and } B\ |\ B\text{ or } B$.
      \item Statements: $S::=\text{skip}\ |\ x:=E\ |\ S_1;S_2\ |\ \text{if }B\text{ then }S_1\text{ else }S_2 |\ \text{while }B\text{ do }S$.
    \end{itemize}
    
    \end{frame}

    \begin{frame}{Semantics domains for while}
      \begin{itemize}
        \item Values: $V=\mathbb{Z}\cup\{\text{true},\text{false}\}$.
        \item Interpretation for constants: $V\to\mathbb{Z}$
        \item States: $\Sigma:\textbf{Var}\to V$, where $\textbf{Var}$ is the set of variables. We denote an update of a state by $\sigma'=\sigma[x\mapsto v]$, which means that $\sigma'(x)=v$ and $\sigma'(y)=\sigma(y)$ for $y\neq x$.
        \item Expression interpretation: $E\to\Sigma\to\mathbb{Z}$. An application of an expression under a state is traditionally written as $val\llbracket E\rrbracket\sigma$.
      \end{itemize}
      The interpretation of an expression under a state is defined by induction on the structure of the expression.
      \begin{itemize}
        \item $val\llbracket n\rrbracket\sigma=n$
        \item $val\llbracket x\rrbracket\sigma=\sigma(x)$
        \item $val\llbracket E_1+E_2\rrbracket\sigma=val\llbracket E_1\rrbracket\sigma+val\llbracket E_2\rrbracket\sigma$
        \item $val\llbracket\text{true}\rrbracket\sigma=\text{true}$
        \item $val\llbracket E_1=E_2\rrbracket\sigma=\text{true}$ if $val\llbracket E_1\rrbracket\sigma=val\llbracket E_2\rrbracket\sigma$ and $\text{false}$ otherwise.
        \item etc.
      \end{itemize}
    \end{frame}

    \begin{frame}{Operational Semantics of while}
      It describes how the execution of while programs is done operationally.
      \vspace*{0.5cm}
      A transition system is a triple $(\Gamma, T,\to)$ where
      \begin{itemize}
        \item $\Gamma$ is a set of configurations. A configuration is a pair $(c,\sigma)$, where $c$ is a command and $\sigma$ is a state.
        \item $T$ is a set of terminal configurations.
        \item $\to\subseteq\Gamma\times\Gamma$ is a transition relation ($\to^*$ is the reflexive transitive closure of $\to$, $\to^+$ is the transitive closure of $\to$).
      \end{itemize}
      \vspace*{0.5cm}
      There are two types of operational semantics:
      \begin{itemize}
        \item Small-step semantics: Describes the execution of a program step by step.
        \item Big-step semantics: Describes the execution of a program in one step.
      \end{itemize} 
      We'll focus on small-step semantics.
      \end{frame}
      
      \begin{frame}{Operational Semantics of while}
        
        \begin{prooftree}
          \AxiomC{}
          \RightLabel{Assign}
          \UnaryInfC{$x:=E\to\sigma[x\mapsto val\llbracket E\rrbracket\sigma]$}
        
        \end{prooftree}
        
        \begin{prooftree}
          \AxiomC{}
          \RightLabel{Skip}
          \UnaryInfC{$(\text{skip},c)\to\sigma$}
        \end{prooftree}
        \begin{columns}
          \column{0.5\textwidth}
        \begin{prooftree}
          \AxiomC{$(c_1,\sigma)\to\sigma'$}
          \RightLabel{Seq1}
          \UnaryInfC{$(c_1;c_2,\sigma)\to(c_2,\sigma')$}
        \end{prooftree}
        \column{0.5\textwidth}
        \begin{prooftree}
          \AxiomC{$(c_1,\sigma)\to(c_1',\sigma')$}
          \RightLabel{Seq2}
          \UnaryInfC{$(c_1;c_2,\sigma)\to(c_1';c_2,\sigma')$}
        \end{prooftree}
      \end{columns}
      \begin{prooftree}
        \AxiomC{}
        \RightLabel{IF for $val\llbracket B\rrbracket\sigma=\text{true}$}
        \UnaryInfC{$(\text{if }b\text{ then }c_1\text{ else }c_2,\sigma)\to(c_1,\sigma)$}
      \end{prooftree}
      \begin{prooftree}
        \AxiomC{}
        \RightLabel{IF for $val\llbracket B\rrbracket\sigma=\text{false}$}
        \UnaryInfC{$(\text{if }b\text{ then }c_1\text{ else }c_2,\sigma)\to(c_2,\sigma)$}
      \end{prooftree}
      \begin{prooftree}
        \AxiomC{}
        \RightLabel{WHILE}
        \UnaryInfC{$(\text{while }B\text{ do }c,\sigma)\to(\text{if }B\text{ then }c;\text{while }B\text{ do }c\text{ else skip},\sigma)$}
      \end{prooftree}
      \end{frame}

      \begin{frame}{Denotational semantics of while}

        While the operational semantics describes the execution of programs in a step-by-step manner via relations, the denotational semantics describes the meaning of programs as a mathematical mapping of states.

        We define $\llbracket\cdot\rrbracket:\Sigma\to\Sigma$ as the denotational semantics of the while language. The denotational semantics of the while language is defined by induction on the structure of the program.
      \end{frame}

      \begin{frame}{Denotational semantics of while}
        \[\llbracket\text{skip}\rrbracket(\sigma)=\sigma\]
        \[\llbracket x:=E\rrbracket(\sigma)=\sigma[x\mapsto val\llbracket E\rrbracket\sigma]\]
        \[\llbracket c_1;c_2\rrbracket=\llbracket c_2\rrbracket\circ\llbracket c_1\rrbracket\]
        \[\llbracket \text{if }b\text{ then }c_1\text{ else }c_2\rrbracket=\llbracket c_1\rrbracket\text{ if }val\llbracket b\rrbracket\sigma=\text{true, otherwise} \llbracket c_2\rrbracket\]
        \[\llbracket \text{while }B\text{ do }c\rrbracket=\llbracket\text{skip}\rrbracket\text{ if }val\llbracket b\rrbracket\sigma=\text{false, otherwise }\llbracket \text{while }B\text{ do }c\rrbracket\circ\llbracket c\rrbracket\]
        \vspace*{0.5cm}

        However, the last definition is not well-defined, as it is not guaranteed that the while loop will terminate. Therefore, we need to define a fixpoint semantics for the while language.
  
      \end{frame}

      \begin{frame}{CPOs}
        A complete partial order (CPO) is a partially ordered set $(M,\leq)$ where
        \begin{itemize}
          \item $M$ has a least element $\bot$.
          \item Each chain $x_1\leq x_2\leq x_3\leq\ldots$ has a least upper bound $\bigsqcup x_i$ in $M$.
          \end{itemize}

        \end{frame}

        \begin{frame}{Least Fixed Point}
          If for a function $M\to M$ a fixed point, i.e., a point $x$ such that $f(x)=x$, exists, then we denote the least fixed point as $\mu_f$, i.e., $x=\mu_f$ if $x$ is a fixed point and $x\leq y$ for all fixed points $y$.
     
          \end{frame}

          \begin{frame}{Knaster-Tarski Theorem}
            For a cpo $(M,\leq)$ and a function $f:M\to M$, the least fixed point $\mu_f$ exists and the following holds:
            \[\mu_f=\bigsqcup_{i\geq 0}f^{(i)}(\bot)\]
          \end{frame}
      
          \begin{frame}{Denotational Semantics for while statement revised}
            We now can define the semantics for $\text{while }B\text{ do }c$ as fixed point of a function $F:(\Sigma\to\Sigma)\to(\Sigma\to\Sigma)$ with 
            \[F(f)(\sigma)=f(\llbracket c\rrbracket(\sigma))\text{ if }val\llbracket b\rrbracket\sigma=\text{true, otherwise }\sigma\]

            \vspace*{0.5cm}
            Note that in our definitions we did not consider the termination of the while loop. 
          \end{frame}
  \end{document}